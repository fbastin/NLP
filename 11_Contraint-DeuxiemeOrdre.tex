\documentclass[usepdftitle=false]{beamer}

\usepackage[utf8]{inputenc}
\usetheme{Singapore}
\usepackage{xcolor}
\setbeamertemplate{footline}[frame number]

% \setbeamercovered{transparent}
%\usecolortheme{crane}
\title[IFT3515]{IFT 3515\\Fonctions à plusieurs variables\\Optimisation avec contraintes\\Conditions d'optimalité -- $2^e$ ordre}
\author[Fabian Bastin]{Fabian Bastin\\DIRO\\Université de Montréal}
\date{}

\usepackage{enumerate}
\usepackage[francais]{babel}

\usepackage{easybmat}
\usepackage{graphicx}

\newtheorem{defn}{Définition}
\newtheorem{lem}{Lemme}
\newtheorem{thm}{Théorème}
\newtheorem{coro}{Corollaire}

\def\red{\color{red}}
\def\blue{\color{blue}}

\def\co{\mathcal{o}}

\def\cA{\mathcal{A}}
\def\cB{\mathcal{B}}
\def\cE{\mathcal{E}}
\def\cI{\mathcal{I}}
\def\cL{\mathcal{L}}
\def\cN{\mathcal{N}}
\def\cR{\mathcal{R}}

\def\bu{\boldsymbol{u}}

\def\cX{\mathcal{X}}

\def\NN{\mathbb{N}}
\def\RR{\mathbb{R}}

\setbeamertemplate{footline}[frame number]

\begin{document}
\frame{\titlepage}

% ------------------------------------------------------------------------------------------------------------------------------------------------------

\begin{frame}
\frametitle{Contexte}

Notes partiellement basées sur \url{http://www.numerical.rl.ac.uk/people/nimg/course/lectures/raphael/lectures/lec10slides.pdf}

\mbox{}

Nous considérons à nouveau le problème d'optimisation nonlinéaire (PNL) général
\begin{align*}
\min_{x \in \RR^n}\ & f(x) \\
\mbox{t.q. } & g_i(x) = 0,\ i\ \in \cE \\
& g_i(x) \leq 0,\ i \in \cI
\end{align*}

Hypothèse: $f$ et $g_i$ ($i \in \cE \cup \cI$) sont des fonctions deux fois continûment différentiables.

\end{frame}

\begin{frame}
\frametitle{Chemin de sortie réalisable: rappel}

\begin{defn}
Soit $x^* \in \RR^n$ un point réalisable pour le problème de PNL et définissons $x \in C^2\left((-\epsilon,\epsilon), \RR^n \right)$ un chemin tel que
\begin{align*}
x(0) &= x^* \\
d &:= \frac{d}{dt}x(0) \ne 0 \\
g_i(x(t)) &= 0,\ i \in \cE,\ t \in (-\epsilon, \epsilon) \\
g_i(x(t)) &\leq 0,\ i \in \cI,\ t \in [0, \epsilon)
\end{align*}
$x(t)$ est {\red un chemin de sortie réalisable} à partir de $x^*$ et le vecteur tangent $d = \frac{d}{dt}x(0)$ est {\red une direction de sortie réalisable} à partir de $x^*$.
\end{defn}

\end{frame}

\begin{frame}
\frametitle{Optimalité au second ordre}

L'analyse d'optimalité au second ordre est basée sur l'observation suivante: si $x^*$ est un minimiseur local du problème de PNL et $x(t)$ est un chemin de sortie réalisable à partir de $x^*$, alors $x^*$ doit aussi être un minimiseur local pour le problème d'optimisation sous contraintes avec une variable
$$
\min_t f(x(t)) \mbox{ t.q. } t \geq 0.
$$

\mbox{}

Analysons d'avantage les directions de sortie réalisables à partir de $x^*$.
La définition implique
$$
d^T \nabla g_i(x^*)
= \frac{d}{dt} g_i(x(t)) |_{t = 0}
= \begin{cases}
\frac{d}{dt} 0 = 0 & i \in \cE \\
\lim_{t \rightarrow 0^+} \frac{g_i(x(t))-0}{t} & i \in \cA(x^*) \cap \cI
\end{cases}
$$

\end{frame}

\begin{frame}
\frametitle{Conditions nécessaires}

Dès lors, les conditions suivantes sont nécessaires pour que $d \in \RR^n$ soit une direction de sortie réalisable à partir de $x^*$:
\begin{align*}
d & \ne 0 \\
d^T \nabla g_i(x^*) &= 0,\ i \in \cE \\
d^T \nabla g_i(x^*) &\leq 0,\ i \in \cA(x^*) \cap \cI
\end{align*}

\end{frame}

%\begin{frame}
%\frametitle{Conditions nécessaires}

%Si $(x^*, \lambda^*)$ satisfont les conditions KKT, le gradient du Lagrangien s'annule:
%$$
%0 = \nabla_x L(x^*,\lambda^*) = \nabla f(x^*) + \sum_{i \in \cI} \lambda_i^* \nabla g_i(x^*) + \sum_{i \in \cE} \lambda_i^* \nabla g_i(x^*)
%$$
%Soit $d \in N^+$. Alors
%\begin{align*}
%0 &= d^T\nabla_x L(x^*,\lambda^*) \\
%& = d^T\nabla f(x^*) + \sum_{i \in \cI} \lambda_i^* d^T\nabla g_i(x^*) + \sum_{i \in \cE} \lambda_i^* d^T\nabla g_i(x^*) \\
%& = d^T\nabla f(x^*) + \sum_{i \in \cA(x^*) \cap \cI} \lambda_i^* d^T\nabla g_i(x^*)
%\end{align*}

%\end{frame}

\begin{frame}
\frametitle{Conditions suffisantes}

D'autre part, si la LICQ tient en $x^*$, du Lemme sur la LICQ, nous avons l'existence d'un chemin de sortie réalisable $x(t)$ à partir de $x^*$ tel que
\begin{align*}
x(0) &= x^* \\
\frac{d}{dt} x(0) &= d \\
g_i(x(t)) &= td^T\nabla g_i(x^*),\ i \in \cE \cup \cA(x^*),\ t \in (-\epsilon, \epsilon)
\end{align*}
Dès lors, lorsque la LICQ tient, les conditions nécessaires sur $d$ sont aussi suffisantes, et fournissent par conséquent une caractérisation exacte pour que $d$ soit une direction de sortie réalisable à partir de $x^*$.

\end{frame}

\begin{frame}
\frametitle{Conditions nécessaires d'optimalité au second ordre}

Soit $x^*$ un minimum local tel que la LICQ tienne.
Les conditions KKT indiquent qu'il existe un vecteur de multiplicateurs de Lagrange $\lambda^*$ tel que
\begin{align*}
\nabla_x L(x^*,\lambda^*) &= 0 \\
g_i(x^*) &= 0,\ i \in \cE \\
g_i(x^*) &\leq 0,\ i \in \cI \\
\lambda_i^* g_i(x^*) &= 0,\ i \in \cI \\
\lambda_i^* &\geq 0,\ i \in \cI
\end{align*}
où
$$
L(x^*, \lambda) = f(x^*) + \sum_{i \in \cE \cup \cI} \lambda_i^* g_i(x^*).
$$

\end{frame}

\begin{frame}
\frametitle{Conditions nécessaires d'optimalité au second ordre}

Soit $x(t)$ un chemin de sortie réalisable à partir de $x^*$ avec la direction de sortie $d$, et considérons le problème restreint
\begin{align*}
\min\ & f(x(t)) \\
\mbox{t.q. } & t \geq 0.
\end{align*} 
Puisque $x^*$ est un minimiseur local du problème de PNL, $t = 0$ doit aussi être un minimiseur local du problème restreint.

Du développement de Taylor autour de $t = 0$ et des conditions KKT,
\begin{align*}
f(x(t)) &= f(x(0)) + (t-0) \frac{d}{dt} f(x(t)) |_{t=0} + O(t^2) \\
&= f(x^*) + t\left(\frac{d}{dt}x(t)|_{t=0}\right)^T \nabla_x f(x(t))|_{t=0} + O(t^2) \\
&= f(x^*) + td^T \nabla f(x^*) + O(t^2) \\
&= f(x^*) - t \sum_{i \in \cE \cup \cI} \lambda_i^* d^T \nabla g_i(x^*) + O(t^2)
\end{align*}

\end{frame}

\begin{frame}
\frametitle{Conditions nécessaires d'optimalité au second ordre}

Nous souhaiterions dès lors montrer que pour $t \geq 0$ assez petit,
$$
- t \sum_{i \in \cE \cup \cI} \lambda_i^* d^T \nabla g_i(x^*) + O(t^2) \geq 0
$$

Notons que
$$
\lambda_i^* d^T\nabla g_i(x^*) = 0,\ i \in \cE \cup (\cI \setminus \cA(x^*))
$$
de sorte que ces termes peuvent être omis.

\mbox{}

Mais qu'en est-il des indices $j \in \cA(x^*) \cap \cI$?
Nous savons déjà que $\lambda_i^* d^T\nabla g_i(x^*) \leq 0,\ i \in \cA(x^*)$.


Nous allons distinguer deux cas.

\end{frame}

\begin{frame}
\frametitle{Conditions nécessaires d'optimalité au second ordre}

Cas 1: il existe un indice $j \in \cA(x^*)$ tel que $\lambda_j^* d^T \nabla g_j(x^*) < 0$.

\mbox{}

Alors, pour tout $0 < t \ll 1$,
\begin{align*}
f(x(t))
&= f(x^*) - t \sum_{i \in \cE \cup \cI} \lambda_i^* d^T \nabla g_i(x^*) + O(t^2) \\
&\geq f(x^*) - t \lambda_j^* d^T \nabla g_j(x^*) + O(t^2) \\
&> f(x^*)
\end{align*}
Dès lors, dans ce cas, $f$ est strictement croissante le long du chemin $x(t)$ pour $t$ suffisamment petit, même si $\frac{d^2}{dt^2} f(x(0))$ était négatif.
% En raison de la contrainte $g_j$, rien ne peut être dit à propos de $\nabla^2 f(x^*) d$.

\end{frame}

\begin{frame}
\frametitle{Conditions nécessaires d'optimalité au second ordre}

Cas 2:
$$
\lambda_i^* d^T \nabla g_i(x^*) = 0,\ i \in \cE \cup \cI. 
$$
Dans ce cas, l'argument précédent ne peut garantir que $f$ est localement croissante le long du chemin $x(t)$.
Nous savons seulement que $\frac{d}{dt} f(x(0)) = 0$, c'est-à-dire que $x^*$ est un point stationnaire du problème restreint.

\mbox{}

Mais il pourrait très bien s'agir d'un maximiseur local ou d'un point selle du problème restreint!

\mbox{}

Les dérivée secondes $\frac{d^2}{dt^2} f(x(0))$ déterminent à présent si $t = 0$ est un minimiseur local du problème restreint, conduisant à une information additionnelle nécessaire.

\end{frame}

\begin{frame}
	\frametitle{Conditions nécessaires d'optimalité au second ordre}
	
Nous noterons
$$
N^+ =
\left\lbrace d \ne 0 \,\bigg|\,
\begin{matrix}
d^T \nabla g_i(x^*) = 0, & i \in \cE \cup \{ j \in \cA(x^*) \cap \cI \,|\, \lambda_j^* > 0 \} \\
d^T \nabla g_i(x^*) \leq 0, & \{ i \in \cA(x^*) \cap \cI \,|\, \lambda_i^* = 0 \} 
\end{matrix}
\right\rbrace
$$
Dès lors, $d \in N^+$ implique $\lambda_i^* d^T \nabla g_i(x^*) = 0$, $i \in \cE \cup \cI$.

%\end{frame}

%\begin{frame}
%\frametitle{Conditions nécessaires d'optimalité au second ordre}

\mbox{}

\begin{thm}[Conditions nécessaires d'optimalité au second ordre]
Soit $x^*$ un minimiseur local du PNL où la LICQ tient.
Soit $\lambda^* \in \RR^m$ un vecteur de multiplicateurs de Lagrange tel que $(x^*, \lambda^*)$ satisfaisant les conditions KKT.
Alors nous avons
$$
d^T \nabla^2_{xx} L(x^*, \lambda^*) d \geq 0
$$
pour tout $d \in N^+$.
\end{thm}

\end{frame}

\begin{frame}
\frametitle{Preuve}

Soit $d \in N^+$ et soit
%. Dès lors,
%$$
%\lambda_i^* d^T \nabla g_i(x^*) = 0, i \in \cE \cup \cI,
%$$
%Soit
$x(\cdot) \in C^2((-\epsilon, \epsilon), \RR^n)$
un chemin de sortie réalisable à partir de $x^*$ correspondant à $d$.
Alors
\begin{align*}
L(x(t), \lambda^*) &=
 f(x(t)) + \sum_{i \in \cE \cup \cI} \lambda_i^* g_i(x(t))\\
&= f(x(t)) + \sum_{i \in \cE \cup \cI} \lambda_i^* t d^T \nabla g_i(x^*) \\
&= f(x(t))
\end{align*}
Le développement de Taylor de $L(x(t),\lambda^*)$ autour de $t = 0$ implique
\begin{align*}
f(x(t)) = &\ L(x^*, \lambda^*) + t \nabla_x L(x^*, \lambda^*)^T d \\
& + \frac{t^2}{2}\left( d^T \nabla_{xx}^2 L(x^*, \lambda^*) d + \nabla_x L(x^*, \lambda^*)^T \frac{d^2}{dt^2}x(t)|_{t=0} \right) \\
& + O(t^3)
\end{align*}

\end{frame}

\begin{frame}
\frametitle{Preuve}

Les conditions KKT impliquent
$$
\nabla_x L(x^*, \lambda^*) = 0
$$
Dès lors
$$
f(x(t)) = L(x^*, \lambda^*) + \frac{t^2}{2} d^T \nabla_{xx}^2 L(x^*, \lambda^*) d + O(t^3)
$$
et par les conditions d'optimalité au premier ordre (dans le cas convexe, la dualité forte), $L(x^*, \lambda^*) = f(x^*)$

\mbox{}

Si $d^T \nabla_{xx}^2 L(x^*, \lambda^*) d < 0$ alors pour $t$ suffisamment petit, $f(x(t)) < f(x^*)$, contredisant l'hypothèse que $x^*$ est un minimiseur local.

\mbox{}

Dès lors, nous devons avoir
$$
d^T \nabla_{xx}^2 L(x^*, \lambda^*) d \geq 0
$$

\end{frame}

\begin{frame}
\frametitle{Conditions suffisantes d'optimalité au second ordre}

\begin{thm}[Conditions suffisantes d'optimalité au second ordre]
Soit $(x^*, \lambda^*) \in \RR^n \times \RR^m$ satisfaisant les conditions KKT, et sous une hypothèse de qualification de contrainte, si
$$
d^T \nabla^2_{xx} L(x^*, \lambda^*)d > 0
$$
pour toute  direction $d \in N^+$.
Alors $x^*$ est un minimiseur local strict.
\end{thm}

\end{frame}

\begin{frame}
\frametitle{Preuve}

Supposons par contradiction que $x^*$ n'est pas un minimiseur local.
Il existe alors une séquence de points réalisables $\{ x_k \}$, $k = 1,\ldots$, telle que
$x_k \ne x^*$ pour tout $k$, $\lim_{k \rightarrow \infty} x_k = x^*$ et
$$
f(x_k) \leq f(x^*),\ \forall k \in \NN.
$$
La séquence
$$
\frac{x_k-x^*}{\| x_k-x^* \|}
$$
tient sur la sphère unité, qui est un ensemble compact, et dès lors nous pouvons extraire une sous-séquence $\{ x_{k_i} \}$, $i \in \NN$, $k_i < k_j$ ($i < j$), tel qu'il existe une direction limite $d := \lim_{k \rightarrow \infty} d_{k_i}$ existe, avec
\begin{align*}
d_{k_i} = \frac{x_{k_i}-x^*}{\| x_{k_i}-x^* \|}
\end{align*}
Puisque $d$ tient sur la sphère unité, nous avons $d \ne 0$.
En ne considérant que la sous-séquence ainsi construite, nous pouvons supposer sans perte de généralité que $k_i = i$.

\end{frame}

\begin{frame}
\frametitle{Preuve}

Vérifions que $d \in N^+$. Considérons tout d'abord les contraintes d'égalité.
$\forall i \in \cE$, en utilisant le développement de Taylor, nous avons
\begin{align*}
0 &= \frac{g_i(x_k)}{\| x_k - x^* \| } \\
&= \frac{1}{\| x_k - x^* \| } \left( g_i(x^*) + \| x_k - x^* \| \nabla g_i(x^*)^Td_k + O(\| x_k - x^* \|^2) \right) \\
&= \nabla g_i(x^*)^Td_k + O(\| x_k - x^* \|)
\end{align*}
En faisant tendre $k$ vers l'infini, nous obtenons
$$
\nabla g_i(x^*)^Td_k = 0
$$

\end{frame}

\begin{frame}
\frametitle{Preuve}

$\forall i \in \cA(x^*) \cap \cI$, en utilisant le développement de Taylor, nous avons de manière similaire
\begin{align*}
0 &\geq \frac{g_i(x_k)}{\| x_k - x^* \| } \\
&= \frac{1}{\| x_k - x^* \| } \left( g_i(x^*) + \| x_k - x^* \| \nabla g_i(x^*)^Td_k + O(\| x_k - x^* \|^2) \right) \\
&= \nabla g_i(x^*)^Td_k + O(\| x_k - x^* \|)
\end{align*}
En faisant tendre $k$ vers l'infini, nous obtenons
$$
\nabla g_i(x^*)^Td \leq 0
$$

\mbox{}

Ainsi, $d$ est une direction de sortie réalisable. Montrons que $d \in N^+$.

\end{frame}

\begin{frame}
\frametitle{Preuve}

Si $d \notin N^+$, alors $\exists j \in \cA(x^*) \cap \cI$ tel que
$$
\lambda_j^* \nabla g_j(x^*)^Td < 0
$$
En vertu du développement de Taylor,
\begin{align*}
\lambda_j^* g_j(x_k)
&= \lambda_j^* g_j(x^*) + \lambda_j^* \| x_k - x^* \| \nabla g_j(x^*)^Td_k + O(\| x_k - x^* \|^2) \\
&= \lambda_j^* \| x_k - x^* \|\nabla g_i(x^*)^Td_k + O(\| x_k - x^* \|^2)
\end{align*}
Dès lors
\begin{align*}
L(x_k, \lambda^*) &= f(x_k) + \sum_{i \in \cE \cup \cI} \lambda_i^* g_i(x_k) \\
&\leq f(x_k) + \lambda_j^* g_j(x_k) \\ 
&=  f(x_k) + \lambda_j^*\| x_k - x^* \|\nabla g_i(x^*)^Td_k + O(\| x_k - x^* \|^2)
\end{align*}

\end{frame}

\begin{frame}
\frametitle{Preuve}

D'autre part, le développement de Taylor de $L(x_k,\lambda^*)$ autour de $(x^*, \lambda^*)$ donne
\begin{align*}
L(x_k,\lambda^*) =\ & L(x^*,\lambda^*) + \| x_k - x^* \| d_k^T \nabla_x L(x^*,\lambda^*)
\\ & + \frac{ \| x_k - x^* \|^2 }{2} d_k^T \nabla^2_{xx} L(x^*,\lambda^*) d_k + O( \| x_k - x^* \|^3 )
\end{align*}
De part les conditions KKT, $\nabla_x L(x^*,\lambda^*) = 0$ et $L(x^*, \lambda^*) = f(x^*)$.
Dès lors, nous obtenons
$$
L(x_k,\lambda^*) = f(x^*) + \frac{ \| x_k - x^* \|^2 }{2} d_k^T \nabla^2_{xx} L(x^*,\lambda^*) d_k + O( \| x_k - x^* \|^3 )
$$

\mbox{}

En combinant ces expressions,
\begin{align*}
f(x_k) & \geq L(x_k, \lambda^*) - \lambda_j^*\| x_k - x^* \|\nabla g_i(x^*)^Td_k + O(\| x_k - x^* \|^2) \\
&= f(x^*) - \lambda_j^*\| x_k - x^* \|\nabla g_i(x^*)^Td_k + O(\| x_k - x^* \|^2)
\end{align*}
et pour $k$ assez grand, $f(x_k) > f(x^*)$ contredisant nos hypothèses de travail.
\end{frame}

\begin{frame}
\frametitle{Preuve}

Ainsi $d \in N^+$, et des hypothèses du théorème, nous avons 
$$
d^T \nabla_{xx}^2 L(x^*, \lambda^*) d > 0.
$$

%Le développement de Taylor de $f(x_k)$ autour de $x^*$ donne

%$$
%f(x_k) = f(x^*) + (x_k-x^*)^T\nabla f(x^*) + O(\| x_k-x^* \|^2)
%$$
%que nous pouvons réécrire, en utilisant la définition de $d_k$,

%$$
%f(x_k) = f(x^*) + \| x_k-x^* \| \nabla f(x^*)^Td_k + O(\| %x_k-x^* \|^2)
%$$
%Par hypothèse, $f(x^*) \geq f(x_k)$, et dès lors%, pour $k$ assez grand,
%$$
%\nabla f(x^*)^Td = \lim_{k \rightarrow \infty} \nabla f(x^*)^Td_k \leq 0
%$$
D'autre part, des conditions KKT et $d \in N^+$,
$$
\nabla f(x^*)^Td = -d^T\sum_{i \in \cE \cup \cI} \lambda_i \nabla g_i(x^*)
= \sum_{i \in \cE \cup \cI} \lambda_i (-d^T \nabla g_i(x^*)) = 0
$$
De plus, de nos hypothèses de départ et des conditions KKT,
$$
f(x^*) \geq f(x_k) \geq f(x_k) + \sum_{i \in \cE \cup \cI} \lambda_i^* g_i(x_k) = L(x_k,\lambda^*)
$$
et, reprenant le développement de $L(x_k,\lambda^*)$
$$
f(x^*) \geq f(x^*) + \frac{ \| x_k - x^* \|^2 }{2} d_k^T \nabla^2_{xx} L(x^*,\lambda^*) d_k + O( \| x_k - x^* \|^3 )
$$
\end{frame}

\begin{frame}
\frametitle{Preuve}

Il s'ensuit
$$
d_k^T \nabla^2_{xx} L(x^*,\lambda^*) d_k \leq O( \| x_k - x^* \| )
$$
en passant à la limite comme $k \rightarrow \infty$
$$
d^T \nabla^2_{xx} L(x^*,\lambda^*) d \leq 0
$$
Ceci contredit l'hypothèse du théorème
$$
d^T \nabla^2_{xx} L(x^*,\lambda^*) d > 0
$$
et donc l'hypothèse d'existence de la séquence $\{ x_k \}$ ne tient pas. Dès lors, $x^*$ est minimiseur local.

\end{frame}

%\begin{frame}
%\frametitle{Preuve}

%Combinant les deux inégalités, nous devons avoir
%$$
%\nabla f(x^*)^Td = 0
%$$

%\mbox{}

%D'autre part, de nos hypothèses de départ et des conditions KKT,
%$$
%f(x^*) \geq f(x_k) \geq f(x_k) + \sum_{i \in \cE \cup \cI} \lambda_i^* g_i(x_k) = L(x_k,\lambda^*)
%$$

%\end{frame}

\begin{frame}
\frametitle{Complémentarité stricte}

Il est courant, et commode, de supposer que la condition de complémentarité stricte s'applique. Celle-ci requiert que le multiplicateur de Lagrange ne peut s'annuler pour une contrainte active, i.e.
$$
\lambda_j^* \ne 0\ \forall j \in \cA(x^*) \cap \cI
$$

\mbox{}

La complémentarité stricte permet de réexprimer $N^+$ de manière très commode. Rappelons que
$$
N^+ =
\left\lbrace d \ne 0 \,\bigg|\,
\begin{matrix}
d^T \nabla g_i(x^*) = 0, & i \in \cE \cup \{ j \in \cA(x^*) \cap \cI \,|\, \lambda_j^* > 0 \} \\
d^T \nabla g_i(x^*) \leq 0, & \{ i \in \cA(x^*) \cap \cI \,|\, \lambda_i^* = 0 \} 
\end{matrix}
\right\rbrace
$$

%Un des avantages de la complémentarité stricte est qu'il est possible d'identifier l'ensemble actif en $x^*$ à proximité de $x^*$. En effet, pour $\epsilon > 0$ suffisamment petit, par continuité, si $x_k \in \cB(x^*, \epsilon)$, $g(x_k) $

sous la complémentarité stricte, nous avons
$$
N^+ =
\left\lbrace d \ne 0 \,|\,
d^T \nabla g_i(x^*) = 0, i \in \cA(x^*)
\right\rbrace
$$

\end{frame}

\begin{frame}
\frametitle{Complémentarité stricte}

Autrement dit, $d$ appartient au noyau du Jacobien des contraintes actives:
$$
d \in Ker(J),\ J = \begin{pmatrix} \nabla g_i(x^*)^T, i \in \cA(x^*) \end{pmatrix}
$$

\mbox{}

Comme
$$
d \in N^+
\Rightarrow
\lambda_i^* d^T \nabla g_i(x^*) = 0,\ i \in \cE \cup \cI,
$$
nous avons également
$$
d \in N^+
\Leftrightarrow
\lambda_i^* d^T \nabla g_i(x^*) = 0,\ i \in \cE \cup \cI,
$$

\end{frame}

\end{document}