\documentclass[usepdftitle=false]{beamer}

\usepackage[utf8]{inputenc}
\usetheme{Singapore}
\usepackage{xcolor}
\setbeamertemplate{footline}[frame number]

% \setbeamercovered{transparent}
%\usecolortheme{crane}
\title[IFT3515]{IFT 3515\\Programmation non-linéaire}
\author[Fabian Bastin]{Fabian Bastin\\DIRO\\Université de Montréal}
\date{}

\usepackage{amsthm}

\def\red{\color{red}}

\def\rit{\mathcal{R}}

\def\cB{\mathcal{B}}
\def\cV{\mathcal{V}}

\usepackage{gnuplottex}
% This variables allows to conditional generate the Gnuplot figures.
\def \gnuplotx {Generate the figures}

\newtheorem{defn}{Définition}

\begin{document}
\frame{\titlepage}

% ------------------------------------------------------------------------------------------------------------------------------------------------------
\begin{frame}
\frametitle{L'objet d'intérêt}

Nous considérons le problème
\[
\min_{x \in S} f(x),
\]
où $f: \rit^n \rightarrow \rit$ et $S$ est un sous-ensemble de $\rit^n$.

\begin{itemize}
\item
$f$: fonction objectif (ou fonction coût)
\item
$S$: ensemble réalisable
\item
$n$: dimension du problème
\end{itemize}

\mbox{}

Tout point $x \in S$ est appelé \textit{solution réalisable} (ou simplement parfois \textit{solution}).

\end{frame}

\begin{frame}
\frametitle{Optimum global}	

Idéalement, nous voudrions trouver un {\red minimum global} de $f$ dans $S$, i.e., un point dans $S$ où la fonction atteint sa plus petite valeur.
Une définition formelle est:
\begin{defn}[Minimiseur global]
Un point $x^*$ est un minimiseur global de $f: \rit^n \rightarrow \rit$ dans $S$ si $f(x^*) \leq f(x)$ pour tout $x \in S$.
\end{defn}

\mbox{}

$x^*$ est appelé {\sl solution optimale} ou {\sl minimiseur de $f$}.
$f(x^*)$ est dite {\sl valeur optimale de $f$}, est nous la noterons parfois $f^*$.

\mbox{}

Un problème d'optimisation n'a pas nécessairement une solution unique.
Il peut ne pas y avoir de solution ou plus d'une solution optimale. Nous définirons
\[
\arg\min f(x) \overset{def}{=} \left\lbrace x \,|\, f(x) \leq f(y)\ \forall y \in S \right\rbrace.
\]

\end{frame}

\begin{frame}
\frametitle{Optimum local}	

Un minimum global peut être difficile à obtenir.
La plupart des algorithmes cherchent seulement un {\red minimum local}, qui est un point qui atteint la plus petite valeur de $f$ dans son voisinage.

\begin{defn}[Minimiseur local]
Un point $x^*$ de $f: \rit^n \rightarrow \rit$ dans $S$ est un minimiseur local s'il existe un voisinage $\mathcal{V}$ de $x^*$ tel que $f(x^*) \leq f(x)$ pour $x \in \mathcal{V} \cap S$.
\end{defn}

Un point qui satisfait cette définition est parfois appelé minimiseur local faible, pour le distinguer d'un minimiseur local strict.

\begin{defn}[Minimiseur local strict]
Un point $x^*$ est un minimiseur local strict (aussi appelé minimiseur local fort) s'il existe un voisinage $\mathcal{V}$ de $x^*$ tel que $f(x^*) < f(x)$ pour tout $x \in \mathcal{V}$ avec $x \ne x^*$.
\end{defn}

\end{frame}

\begin{frame}
\frametitle{Optimum local isolé}	

\begin{defn}[Minimiseur local isolé]
Un point $x^*$ est un minimiseur local isolé s'il existe un voisinage $\mathcal{V}$ de $x^*$ tel que $f(x^*) < f(x)$ pour tout $x \in \mathcal{V}$ avec $x \ne x^*$ et il n'y a pas d'autre minimiseur local dans $\mathcal{V}$.
\end{defn}

\mbox{}

Un minimiseur local strict n'est pas toujours isolé, mais tout minimiseur local isolé est strict.

\end{frame}

\begin{frame}[fragile]
\frametitle{Exemple}

(Tiré de Nocedal et Wright)
Considérons la fonction suivante:
\[
f(x) = x^4 \cos{\frac{1}{x}} +2x^4.
\]

\mbox{}

$f(x) \in C^2$ (deux fois continûment différentiable) et a un minimum local strict en $x^* = 0$.
Toutefois, tout voisinage de $x^*$ contient d'autres minimums locaux stricts.
%, et nous pouvons extraire une séquence telle que $x_n \rightarrow 0$, comme $n \rightarrow \infty$.

%\ifx \gnuplotx \undefined
%\else
\begin{center}
	\begin{gnuplot}[terminal=cairolatex,scale=0.5]
plot [-0.1:0.1][0:0.00015]x*x*x*x*(cos(1/x)+2) title '$x^4\left(\cos\left(\frac{1}{x}\right)+2\right)$'
	\end{gnuplot}
\end{center}
%\fi

\end{frame}

\begin{frame}
\frametitle{Voisinage}

Un voisinage $\cV(x)$ d'un point $x$ est un ensemble ouvert contenant $x$.

\mbox{}

Typiquement, $\cV(x)$ est pris comme un boule ouverte centrée en $x$, de rayon $\epsilon > 0$:
$$
\cB(x,\epsilon) = \lbrace y \,|\, \| x-y \| < \epsilon\rbrace
$$
Le choix de la norme n'est pas spécifié, mais la plupart du temps, nous travaillerons avec la norme 2
$$
\|x-y\|_2 = \sqrt{ (x-y)^T(x-y) }
$$

\mbox{}

Quel que soit $\cV(x)$, $\exists \epsilon > 0$ tel que $\cB(x,\epsilon) \subseteq \cV$.
Nous ne perdons dès lors aucune généralité en travaillant avec une boule ouverte centrée en $x$.

\end{frame}

\begin{frame}
\frametitle{Ensemble réalisable}

Généralement, l'ensemble réalisable $S$ sera représenté sous forme de contraintes d'égalités et d'inégalités:
\begin{align*}
& c_i(x) = 0,\ i \in \mathcal{E}, \\
& c_i(x) \leq 0,\ i \in \mathcal{I}.
\end{align*}

\mbox{}
	
\begin{itemize}
\item 
$\mathcal{E}$: ensemble des contraintes d'égalité
\item 
$\mathcal{I}$: ensemble des contraintes d'inégalité
\end{itemize}

\mbox{}

En résumé, nous considérerons des problèmes de la forme
\begin{align*}
\min_x\ & f(x) \\
\mbox{t.q. } & c_i(x) = 0,\ i \in \mathcal{E}, \\
& c_i(x) \leq 0,\ i \in \mathcal{I}.
\end{align*}

\end{frame}

\begin{frame}
\frametitle{Reformulations}

La formulation précédente est assez générale, et couvre d'autres variantes par simple reformulation.

\mbox{}

Nous pouvons par exemple considérer un problème de maximisation avec la transformation
\[
\max_{x \in S} f(x) = -\min_{x \in S} (-f(x))
\]

\mbox{}

Les problèmes sont aussi définis à une constante près, comme
\[
\min_x (f(x)+\kappa) = \kappa + \min_x f(x)
\]

\end{frame}

\begin{frame}
\frametitle{Reformulations des contraintes}

Nous pourrions aussi ne considérer que des contraintes d'égalité, en introduisant des variables d'écart:
\begin{align*}
c_i(x) & \leq 0 \\
& \Leftrightarrow \\
c_i(x) + s_i & = 0, \quad s_i \geq 0 \\
\end{align*}
Similairement, nous pourrions juste écrire des contraintes d'inégalité:
\begin{align*}
c_i(x) & = 0 \\
& \Leftrightarrow \\
c_i(x) & \leq 0, \\
-c_i(x) & \leq 0
\end{align*}

\end{frame}

\begin{frame}
\frametitle{Structure générale d'un algorithme d'optimisation}

On fixe un point de départ $x_0$ et on génère une séquence de points $x_1$, $x_2$,\ldots sur le principe suivant.
Poser $k = 0$, et calculer $f(x_0)$. Tant qu'une condition d'arrêt n'est pas satisfaite, répéter
\begin{enumerate}
\item
Poser $k := k+1$.
\item
Déterminer une solution candidate $x_{k+1}$, et calculer $f_(x_{k+1})$.
\end{enumerate}

\mbox{}

$k$ est appelé {\sl indice d'itération}.
On voudrait que le nombre d'itérations soit fini, et si ce n'est pas le cas, que la séquence de point converge vers une solution, i.e.
\[
\exists\, x^* \mbox{ tel que } \lim_{n \rightarrow \infty} x_n = x^*,
\]
et
\[
x^* = \arg\min f(x). 
\]

\end{frame}

\begin{frame}
\frametitle{Convergence}

La convergence peut être testée de différentes manières, dépendamment du problème et de la technique d'optimisation utilisée.
Par exemple, on peut décider de s'arrêter quand la réduction de la fonction objectif entre deux itérations devient négligeable:
\[
|\Delta f| = |f(x_{k+1}) - f(x_k)| \leq \epsilon,
\]
où $\epsilon$ est la {\red tolérance d'optimisation} pour la fonction objectif $f$.

\mbox{}

Alternativement, on peut s'arrêter lorsque le changement entre les itérés devient non significatif:
\[
\| x_{k+1} - x_k \|_k \leq \epsilon,
\]
où $\| \cdot \|_k$ est une certaine norme (pouvant théoriquement dépendre de l'indice d'itération $k$).

\end{frame}

\begin{frame}
\frametitle{Conditions d'optimalité}

Supposons que nous nous arrêtions à l'itération $\ell$.
Ceci garantit-il que $x_{\ell}$ est une solution du problème d'optimisation?

\mbox{}

Pas nécessairement!

\mbox{}

Nous voudrions pouvoir établir des {\red conditions d'optimalité}, qui peuvent être vérifiées, et telles que si celles-ci sont remplies, $x_{\ell}$ soit solution.

\mbox{}

En pratique, celles-ci ne seront jamais satisfaites, mais nous voudrions pouvoir détecter que $x_{\ell}$ est ``quasi''-optimal, dans le sens où
\[
f(x_{\ell}) \leq f(x)+\delta
\]
pour tout $x \in S$, et $\delta > 0$ petit.

\end{frame}

\begin{frame}
\frametitle{Pas de méthode miracle}

\begin{quote}
[\ldots] for any algorithm, any elevated performance over one class of problems is offset by performance over another class
\end{quote}
{\sl No free lunch theorems for optimization}, David H. Wolpert and William G. Macready, IEEE Transactions on Evolutionary Computation 1(1), pp. 67--82, 1997.

\end{frame}

\begin{frame}
\frametitle{Avertissement}

Beaucoup de fausses idées circulent sur les propriétés et les algorithmes d'optimisation.
Plusieurs sont recensées dans le document ``Myths and Counterexamples in Mathematical Programming'', disponible à l'adresse
\url{http://glossary.computing.society.informs.org/myths/CurrentVersion/myths.pdf}

\mbox{}

Nous tâcherons d'éviter ces pièges dans le cours, mais vous êtes invités à me corriger au besoin!

\end{frame}

\end{document}

