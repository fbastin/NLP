\documentclass[t,usepdftitle=false]{beamer}

\usepackage[utf8]{inputenc}
\usetheme{Warsaw}
\usepackage{xcolor}
% \setbeamercovered{transparent}
%\usecolortheme{crane}
\title[IFT3515]{IFT 3515\\Fonctions à plusieurs variables\\Optimisation avec contraintes\\Gradient conjugué projeté}
\author[Fabian Bastin]{Fabian Bastin\\DIRO\\Université de Montréal}
\date{Hiver 2017}

\usepackage{enumerate}
\usepackage[francais]{babel}

\usepackage{easybmat}
\usepackage{graphicx}

\newtheorem{defn}{Définition}
\newtheorem{lem}{Lemme}
\newtheorem{thm}{Théorème}
\newtheorem{coro}{Corollaire}

\def\red{\color{red}}
\def\blue{\color{blue}}

\def\co{\mathcal{o}}

\def\cB{\mathcal{B}}
\def\cN{\mathcal{N}}
\def\cR{\mathcal{R}}

\def\bu{\boldsymbol{u}}

\setbeamertemplate{footline}[frame number]

\begin{document}
\frame{\titlepage}

% ------------------------------------------------------------------------------------------------------------------------------------------------------

\begin{frame}
\frametitle{Méthode du gradient conjugué projeté}

Il est possible d'adapter l'algorithme du gradient conjugué tronqué pour tenir compte de contraintes de bornes, comme expliqué dans Lin et Moré, 1999.

\mbox{}

Idée de base: appliquer l'algorithme du gradient conjugué jusqu'à ce qu'une contrainte soit rencontrée, auquel cas l'ensemble actif est mis à jour et on recommence les itérations du gradient conjugué dans l'espace réduit (les contraintes actives restent actives).

\end{frame}

\end{document}